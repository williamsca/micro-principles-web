
%%%%%%%%%%%%%%%%%%%%%%%%%%%%%%%%%%%%%%%%%%%%%%%%%%%%%%%%%%%%%%%
%
% Welcome to Overleaf --- just edit your LaTeX on the left,
% and we'll compile it for you on the right. If you open the
% 'Share' menu, you can invite other users to edit at the same
% time. See www.overleaf.com/learn for more info. Enjoy!
%
%%%%%%%%%%%%%%%%%%%%%%%%%%%%%%%%%%%%%%%%%%%%%%%%%%%%%%%%%%%%%%%
\documentclass{exam}

\lhead{ECON 2010}
\chead{Practice}
\lfoot{11/29/2022}
\rhead{Fall 2022}

% \printanswers
\noprintanswers

\begin{document}

\section{Practice Questions}

\begin{questions}

\question (Final - 2010Q21) When interest rates rise, 

\begin{choices}
    \choice the ‘price’ of borrowing money goes down.
    \choice bond prices go up.
    \CorrectChoice bond prices fall.
    \choice negative externalities go away.
    \choice the reaction of bond prices is indeterminate.
\end{choices}

\begin{solution} The answer is \textbf{C}. A higher interest rate means that money in the future is ``cheaper'' than it used to be. Bonds are just promises to pay money in the future, so bond prices fall. \end{solution}

\question (Final - 2010Q4) As price falls along a given demand curve for pretzels,
\begin{choices}
\choice quantity demanded, total utility, consumer surplus and marginal utility increase; consumer
expenditure might increase, decrease, or remain constant.
\choice quantity demanded, total utility, and consumer surplus increase; marginal utility and consumer
expenditure decrease.
\choice quantity demanded, total utility, consumer surplus, and consumer expenditure increase; marginal
utility decreases.
\CorrectChoice quantity demanded, total utility, and consumer surplus increase; marginal utility decreases;
consumer expenditure might increase, decrease, or remain constant.
\choice quantity demanded, total utility, marginal utility, consumer surplus, and consumer expenditure all
increase.
\end{choices}

\begin{solution} The answer is \textbf{D}. Marginal utility falls because of the Principle of Diminishing Marginal Utility. Consumer expenditures are indeterminate: because $Q$ increases and $P$ decreases, the effect on total expenditure $PQ$ will depend on the elasticity of demand.\end{solution}

\question (Final - 2010Q20) Gilligan runs the only retail store selling apple pies on an island. If the cost of apples falls, he can increase
profits by 
\begin{choices}
\choice raising the price of his apple pies.
\choice charging the highest price he can for an apple pie.
\choice using fewer apples in each pie.
\CorrectChoice lowering the price of his pies.
\choice charging a price for each pie equal to marginal cost.
\end{choices}

\begin{solution} Gilligan will pass some part of his cost savings onto consumers. \end{solution}

\question (Final - 2010Q42) If the value of the price elasticity of demand is -0.2, this means that a
\begin{choices}
\choice 5 percent increase in price causes a 1 percent decrease in demand.
\choice 0.2 percent decrease in price causes a 1 percent increase in quantity demanded.
\CorrectChoice 5 percent decrease in price causes a 1 percent increase in quantity demanded.
\choice 0.2 percent decrease in price causes a 0.2 percent increase in quantity demanded.
\choice 100 percent decrease in price causes a 200 percent increase in quantity demanded.
\end{choices}

\begin{solution} Answer $\textbf{C}$ satisfies the definition of elasticity: $$\epsilon = \frac{\%\Delta Q}{\%\Delta P} = -0.2 = \frac{1}{-5}$$

Elasticity measures the change in \textit{quantity} demanded, not the change in demand, so A is incorrect.
\end{solution}

\question (Final - 2010Q75) Which of the following is correct regarding the labor supply curve?
\begin{choices}
\choice As the wage rate increases, the opportunity cost of working decreases
\choice As the wage rate increases, the substitution effect can overcome the income effect after a certain
wage rate is reached, causing the labor supply curve to "bend backwards." 
\CorrectChoice As the wage rate increases, the income effect can overcome the substitution effect after a certain
wage rate is reached, causing the labor supply curve to ``bend backwards." 
\choice Both a. and b. are correct. 
\choice Both a. and c. are correct. 
\end{choices}

\begin{solution} Answer A is incorrect. The opportunity cost of working is the value of the next best alternative, which we typically think of being either leisure or home production. A higher wage does not affect the value of not working.

The income effect is what causes the labor supply curve to ``bend backwards'' because more income leads people to purchase more leisure and work less, so the answer is \textbf{C}. \end{solution}

\question (Final - 2010Q19) If a monopolist engages in perfect price discrimination,
\begin{choices}
\choice the marginal revenue curve becomes steeper.
\CorrectChoice the demand curve becomes  the firm's marginal revenue curve.
\choice the demand curve is steeper than the firm's marginal revenue curve.
\choice the demand curve is not as steep as the firm's marginal revenue curve.
\choice there is no way to define the firm's marginal revenue.
\end{choices}

\question (Final - 2010Q49) Which of the following could explain a decrease in the demand for labor in a particular job?
\begin{choices}
\choice Additional training that increases the productivity of each unit of labor in this market 
\choice An increase in the amount of risk associated with this job 
\choice A decrease in the amount of risk associated with this job 
\choice An improvement in the working conditions associated with this job 
\CorrectChoice A decrease in the productivity of each unit of labor in this market 
\end{choices}

\begin{solution} The answer is \textbf{E}. Answers A-D describe factors that would affect the \textit{supply} of labor, not the demand. \end{solution}

\end{questions}

\end{document}
