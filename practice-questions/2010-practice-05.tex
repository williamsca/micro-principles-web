\documentclass{exam}

\usepackage{graphicx}
\graphicspath{ {./images/} }

\lhead{ECON 2010}
\chead{Practice}
\lfoot{9/21/2023}
\rhead{Fall 2023}

\printanswers
% \noprintanswers

\begin{document}

\section{Practice Questions}

\begin{questions}

\question (2020Q8) In a famous antitrust case, an economist was asked to explain the difference between "economic cost" and "accounting cost." This example was given (which is paraphrased below):

\begin{blockquote}
So, for example, I spent some money to dig a gold mine, and it turns out I really discovered a rich vein of gold. My accounting costs of the gold mine might be very low, but the economic cost to me of operating the gold mine might be very high.
\end{blockquote}

Explain why or how this might be the case. 

\begin{solution}
There are opportunity costs of operating the mine. Most obviously, the economist's time could be spent consulting or publishing research. There is also an opportunity cost associated with the land. If the gold mine is located under Manhattan, operating the mine would mean losing out on substantial rents that could be obtained by leasing the land out.
\end{solution}

\question (2020Q11) To prevent “price gouging” on toilet paper at the start of the COVID-19 pandemic, several states placed a price ceiling on this product. Retail stores were ordered not to raise the price of toilet paper above the level that prevailed prior to the pandemic. The demand for toilet paper at retail went up as a result of the pandemic – in part because many people were not allowed to go to work (or travel); they were confined to their homes and therefore used more toilet paper than before.

\begin{parts}
\part Based on the demand and supply model, what are two consequences you would predict would follow as a result of the price ceiling?

\begin{solution}
A shortage; a black market; allocation by sellers' preferences; allocation by lines
\end{solution}

\part One state official said what was needed was a price floor, rather than a price ceiling, one that also would be set at the level that prevailed prior to the pandemic. Based on the demand and supply model, what would be one economic effect of a price floor on toilet paper?

\begin{solution}
The price floor would have no effect.
\end{solution}

\end{parts}

\question (2020Q9) New Orleans, sometimes called the Crescent City, is a port; it also is a major tourist destination. Imagine in your mind’s eye a Production Possibility Curve ("PPC") for New Orleans that is concave from below. Imagine that the PPC has tourist activity on one axis and shipping on the other. The city is "hit" by a hurricane which leads to flooding and major damage to the port and downtown.

\begin{parts}
\part Describe what the effect of the hurricane will be on the PPC. 

\begin{solution}
    The PPC will shift inward.
\end{solution}

\part Why would a PPC be concave from below?

\begin{solution}
    The resources which are used to produce tourism (like hotels and taxi drivers) are not perfect substitutes for the resources used to produce shipping (like port workers and truck drivers).
\end{solution}
\end{parts}

\question (2018Q7) A prominent law professor at the University of Pennsylvania authored an article in the Fordham Law Review on economic welfare. He wrote: “Perfect complements are goods that are invariably used together – or, more technically, situations in which one good has no value unless it can be consumed together with the other good.”

\begin{parts}
    \part Is this law professor defining the economic concept of a complementary good correctly?

    \begin{solution}
        Yes
    \end{solution}

    \part Provide your own example of two goods that, in the taxonomy of economics, are complements.

    \begin{solution}
        Peanut butter and jelly
    \end{solution}

    \part If your example of two complementary goods is correct, what can we say about the cross-elasticity of demand between the two goods you selected?

    \begin{solution}
        The cross-elasticity of demand will be negative.
    \end{solution}
    

\end{parts}

\question (2019Q3) Assume you are one of the students in the Econ 201 Guestbook who wants to do market research someday. Upon graduation, Procter \& Gamble hires you to be on the management team engaged in marketing the company’s brand of peanut butter, Jif.

\begin{parts}

    \part The first question you are asked is, "What are some of the variables that affect the 'demand' for Jif?" How do you respond?

    \begin{solution}
        Consumer's tastes and incomes; advertising; the price of related goods
    \end{solution}

    \part You are asked to come up with an estimate of the price elasticity of demand for Jif. Before you even start, you want to be clear on the concept: so, what is the definition of price elasticity of demand?

    \begin{solution}
        The percentage change in quantity demanded divided by the percentage change in price.
    \end{solution}

    \part What would cause a product like Jif to have a high elasticity of demand versus a low elasticity of demand?

    \begin{solution}
        If there are many substitutes for Jif, the elasticity of demand will be high. If there are few substitutes for Jif, the elasticity of demand will be low.
    \end{solution}

    \part Your boss asks you this question: would the elasticity of demand for Jif be higher or lower than the elasticity of demand for all peanut butter? What is your answer – and why?

    \begin{solution}
        The elasticity of demand for Jif will be higher than the elasticity of demand for all peanut butter. Jif is a brand of peanut butter. If the price of Jif increases, consumers can switch to other brands of peanut butter. If the price of all peanut butter increases, consumers cannot switch to other brands of peanut butter.
    \end{solution}

    \part Maddie, one of the others on the market research team with you, says, “If we knew the income elasticity of demand for Jif, we could tell if the good is normal or inferior.” What does Maddie mean by this?

    \begin{solution}
        If the income elasticity of demand is positive, Jif is a normal good. If the income elasticity of demand is negative, Jif is an inferior good.
    \end{solution}

    \part Your boss then says, “I think a lot of our customers use Jif to make PBJ sandwiches” (as in, “peanut butter and jelly” sandwiches). If your boss is correct, how would you frame the problem of figuring out the cross-elasticity of demand between Jif peanut butter and jelly?

    \begin{solution}
        The cross-elasticity of demand between Jif peanut butter and jelly is the percentage change in the quantity of Jif peanut butter demanded divided by the percentage change in the price of jelly. To calculate this, we would need data on how the quantity of Jif peanut butter demanded changes when the price of jelly changes.
    \end{solution}

\end{parts}

\end{questions}

\end{document}