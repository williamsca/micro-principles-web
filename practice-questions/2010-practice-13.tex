\documentclass{exam}

\usepackage{graphicx}
\graphicspath{ {./images/} }

\lhead{ECON 2010}
\chead{Practice}
\lfoot{10/19/2023}
\rhead{Fall 2023}

%\printanswers
 \noprintanswers

\begin{document}

\section{Practice Questions}

\begin{questions}

\question (Ch. 8.7) There's a gas shortage in Gasland. You're presented with two proposals that will achieve the same level of reduction in the use  of gas. Proposal A would force everybody to reduce their gas consumption by 5\%. Proposal B would impost a 50-cent tax on the consumption of a gallon of gas, which would also achieve a 5 \% reduction. Consumers can be divided into two groups: one group whose demand is elastic, and another whose demand is inelastic.
\begin{enumerate}
\item How will the proposals affect each group?
\begin{solution} Proposal A would force a downward shift in each demand curve by reducing quantity at every price. Proposal B would raise the price at each quantity, also shifting the demand curves down. \end{solution}
\item Which group would support the regulatory policy (A)? Which would support the tax policy (B)?
\begin{solution} The consumers with more elastic demand can more easily adjust their usage and would therefore favor Proposal A. Those with inelastic demand are more likely to favor the tax. \end{solution}

\end{enumerate}

\vspace{15em}

\question (Ch. 8.8) Economists studied the effect of Charlottesville, VA moving from charging a flat fee for garbage collection to charging \$0.80 per 32-gallon bag and found the following: the \textbf{weight} of garbage collected fell by 14\%; the \textbf{volume} of garbage collected fell by 37\%; and the weight of recycling rose by 16\%.
\begin{enumerate}
\item Why did recycling increase and garbage collection decrease?
\begin{solution} Recycling and garbage collection are *substitutes*. The price of garbage collection rose, so demand for recycling increased. \end{solution}

\item Why did the weight fall by less than the volume?
\begin{solution} The fee is based on volume, so people started to squeeze more weight into the same 32-gallon bag. \end{solution}

\item Demonstrate, using supply and demand curves, the effect of the change in pricing on the volume of garbage collected.
\begin{solution} Under a flat fee, the marginal cost of garbage disposal was zero. Under volume pricing, the price is now fixed at .8, so volume falls. \end{solution}

\end{enumerate}

\end{questions}

\end{document}