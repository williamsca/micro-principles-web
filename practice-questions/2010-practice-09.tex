
%%%%%%%%%%%%%%%%%%%%%%%%%%%%%%%%%%%%%%%%%%%%%%%%%%%%%%%%%%%%%%%
%
% Welcome to Overleaf --- just edit your LaTeX on the left,
% and we'll compile it for you on the right. If you open the
% 'Share' menu, you can invite other users to edit at the same
% time. See www.overleaf.com/learn for more info. Enjoy!
%
%%%%%%%%%%%%%%%%%%%%%%%%%%%%%%%%%%%%%%%%%%%%%%%%%%%%%%%%%%%%%%%
\documentclass{exam}

\lhead{ECON 2010}
\chead{Practice}
\lfoot{12/05/2022}
\rhead{Fall 2022}

\printanswers
% \noprintanswers

\begin{document}

\section{Practice Questions}

\begin{questions}

\question (Final - 2010Q7) Consider two resource markets in which the demand curves slope downward. In market A, the supply curve is horizontal, 
equilibrium price is \$6, and 100 units of the resource are hired. In market B, the supply curve is vertical, equilibrium price is \$20, and 30 units of the resource are hired. Which of the following is true?
\begin{choices}
\CorrectChoice Total resource earnings are the same in both markets.
\choice Total resource earnings are greater in market A.
\choice Total resource earnings are greater in market B.
\choice There is more economic rent in market A.
\choice There is derived demand in market A, but not in market B.
\end{choices}

\question (Final - 2010Q11) The market for bubble gum is competitive with a current equilibrium price of 50 cents and quantity of 100,000 units. Which of the following events would lead to a new equilibrium price of 60 cents and quantity of 90,000 units? 
\begin{choices}
\choice An increase in the price of other kinds of gum and candy.
\CorrectChoice An increase in the price of the ingredients used to make bubble gum.
\choice A decrease in the number of young people in the population.
\choice An agreement by workers in the bubble gum industry to work for lower wages.
\choice An increase in income (assuming bubble gum is a normal good)
\end{choices}

\question (Final - 2010Q19) If a monopolist engages in perfect price discrimination,
\begin{choices}
\choice the marginal revenue curve becomes steeper.
\CorrectChoice the demand curve becomes  the firm's marginal revenue curve.
\choice the demand curve is steeper than the firm's marginal revenue curve.
\choice the demand curve is not as steep as the firm's marginal revenue curve.
\choice there is no way to define the firm's marginal revenue.
\end{choices}

\question (Final - 2010Q49) Which of the following could explain a decrease in the demand for labor in a particular job?
\begin{choices}
\choice Additional training that increases the productivity of each unit of labor in this market 
\choice An increase in the amount of risk associated with this job 
\choice A decrease in the amount of risk associated with this job 
\choice An improvement in the working conditions associated with this job 
\CorrectChoice A decrease in the productivity of each unit of labor in this market 
\end{choices}

\begin{solution} The answer is \textbf{E}. Answer A would \textit{increase} demand for labor, not decrease it. Answers B-D describe factors that would affect the \textit{supply} of labor, not the demand. \end{solution}

\question (Final - 2010Q71) Suppose that the only firm selling a particular type of women's apparel exits the industry because demand is too low. The correct analysis of this situation is that 
\begin{choices}
\choice the firm’s decision is irrational, since monopolies are not limited by the demand curve
\choice the firm’s decision is irrational, since monopolies never go out of business
\choice the firm’s decision is irrational, since it could simply raise the price
\choice the price of the firm’s product was lower than the marginal cost in the long run
\CorrectChoice the price of the firm’s product was lower than the average total cost in the long run 
\end{choices}

\begin{solution} Answers A, B, and C are incorrect: a monopolist \textbf{is} limited by the demand curve for her product. Unlike a competitive firm, the monopolist is allowed to choose a price and quantity along the demand curve that maximizes her revenue. Answer D is incorrect because the optimal price is always higher than the marginal cost; otherwise, the monopolist is losing revenue on the sale. \end{solution}

\end{questions}

\end{document}
