%%%%%%%%%%%%%%%%%%%%%%%%%%%%%%%%%%%%%%%%%%%%%%%%%%%%%%%%%%%%%%%
%
% Welcome to Overleaf --- just edit your LaTeX on the left,
% and we'll compile it for you on the right. If you open the
% 'Share' menu, you can invite other users to edit at the same
% time. See www.overleaf.com/learn for more info. Enjoy!
%
%%%%%%%%%%%%%%%%%%%%%%%%%%%%%%%%%%%%%%%%%%%%%%%%%%%%%%%%%%%%%%%
\documentclass{exam}

\lhead{ECON 2010}
\chead{Practice}
\lfoot{9/7/2023}
\rhead{Fall 2023}

% \printanswers
\noprintanswers

\begin{document}

\section{Practice Questions}

\begin{questions}

\question (2022Q6). As a consequence of the war between Russia and Ukraine:

\begin{parts}
    \part What is the probable effect on the retail price of consumer products in Russia sold by U.S. and European retailers (like McDonalds) who have pulled their operations out of Russia and no longer do business there?
    \begin{solution}
        The price of consumer products will increase.
    \end{solution}
    \part What is the probable effect on the market price of armaments and military equipment made in
    the U.S. and Europe?
    \begin{solution}
        The price of armaments and military equipment in the US and Europe will increase.
    \end{solution}

    \part In the space below draw a diagram that illustrates the effect of the war on the Production Possibility Curve (or frontier) for Ukraine, where Consumption Goods are on the Y axis and Capital Goods are on the X axis.
    \begin{solution}
        A graph with a convex PPF shown to be shifting inwards.
    \end{solution}
\end{parts}

\question (2019Q10) The founder and chief executive officer (CEO) of a well-known company has been accused of sexual harassment of employees. He refuses to resign from the company, provoking a call for a boycott of the products produced by this company. A substantial number of people join the boycott. Based on the demand and supply model, what is likely to happen to the demand for goods of companies who produce the same products as the company being boycotted?
\begin{solution}
    Demand for the goods of the boycotted company's competitors will increase. Consumers still want to consume this type of product, so they will substitute away from the boycotted company's products and towards the products of its competitors.
\end{solution}

\question (2019Q11) The poaching (illegal killing) of elephants for the valuable ivory in their tusks has threatened the population of elephants in some parts of Africa.  In response, Kenya generally prohibits the hunting of elephants.  Botswana, on the other hand, welcomes hunters to shoot elephants if the hunters have purchased a license.  Some observers are puzzled to learn that elephant herds are growing in Botswana and shrinking in Kenya. From an economic perspective, what might explain this economic ``story''? 

\begin{solution}
The license to hunt an elephant is a property right: the owner can use it themselves, exclude other hunters from it, and sell it. This creates incentives for both private industry and the government to protect elephants.

First, businesses can profit from the sale of hunting licenses by providing complementary goods: guides, guns, ammunition, etc. These businesses are incentivized to protect the elephant population so that they can continue to sell their services.

Second, the government can earn revenue from the sale of hunting licenses. If the elephant population falls too low, though, they will lose this revenue, so they have a financial incentive to limit the number of licenses sold to a sustainable level.

In Kenya, protecting elephants from poaches is a drain on the government's resources, and private businesses can't legally profit from providing services to hunters. Thus, neither group has an incentive to protect the elephant population.
\end{solution}


\end{questions}


\end{document}
