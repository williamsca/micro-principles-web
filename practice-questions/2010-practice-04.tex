%%%%%%%%%%%%%%%%%%%%%%%%%%%%%%%%%%%%%%%%%%%%%%%%%%%%%%%%%%%%%%%
%
% Welcome to Overleaf --- just edit your LaTeX on the left,
% and we'll compile it for you on the right. If you open the
% 'Share' menu, you can invite other users to edit at the same
% time. See www.overleaf.com/learn for more info. Enjoy!
%
%%%%%%%%%%%%%%%%%%%%%%%%%%%%%%%%%%%%%%%%%%%%%%%%%%%%%%%%%%%%%%%
\documentclass{exam}

\lhead{ECON 2010}
\chead{Practice}
\lfoot{10/6/2022}
\rhead{Fall 2022}

\printanswers
% \noprintanswers

\begin{document}

\section{Practice Questions}

\begin{questions}
\question On the front page of the 9.23.19 edition of the newspaper USA Today one reads: \textit{``Dairy Farmers weigh drop in milk drinkers. Consumption declines in favor of soy, almond and other ‘fake’ milks."}  Assume the market for cows’ milk produced by dairy farmers approximates the model of perfect competition. 
\begin{parts}
\part What is the likely effect of the “drop” in the number of “milk drinkers” on the market price of milk and the market quantity?

\begin{solution}
Market price and quantity decrease due to lower demand for milk.
\end{solution}

\part If the market demand for milk is elastic, what is the likely effect of the “drop in milk drinkers” on the total revenue going to milk producers (i.e., dairy farmers)?

\begin{solution}
Total revenue decreases. The point about the elasticity of demand is a red herring. We know that $TR = PQ$. Given that both $P$ and $Q$ are lower, total revenue must fall.
\end{solution}

\part Given this trend in milk consumption, what is the likely effect on the number of milk producers?

\begin{solution}
The number of milk producers decreases.
\end{solution}

\part What is the likely effect of the “drop in milk drinkers” on the price of yogurt products (cows’ milk being an input for yogurt production)?

\begin{solution}
A lower price of milk makes yogurt cheaper to produce, increasing the supply. Consequently, the price of yogurt products falls. 
\end{solution}

\part If the U.S. Department of Agriculture determines that milk prices are “too low” to be fair for dairy farmers and “supports” the price of milk at a level higher than the market clearing (or equilibrium) price, what is the economic consequence of the price support?
\begin{solution}
A price support -- i.e., a price floor -- set above the equilibrium price will cause a surplus.
\end{solution}

\part What are two policy measures the government might take to make the milk price support effective?
\begin{solution}
The government could buy the surplus milk and destroy it, convert it to non-perishable foods like cheese or yogurt, or distribute it to the domestic or foreign poor. The government could also pay dairy farmers not to produce milk.
\end{solution}

\part What is the likely effect of the drop in the consumption of cows’ milk on the price of what the article calls “soy, almond and other ‘fake’ milks”?
\begin{solution}
The price of soy, almond, and other `fake' milks increases because a change in consumer tastes increased demand for those products.
\end{solution}

\end{parts}

\question (2018Q1) The Colander textbook claims that a tax increase on gasoline will have different effects depending on whether the tax is imposed only in Washington, D.C. or across the entire U.S. The author claims this is because of different elasticities. Would the demand for gasoline in Washington, D.C., or the entire nation, have the more elastic demand? Why?

\begin{solution}

    D.C. has more elastic demand because consumers can more easily substitute to purchasing gas outside of D.C. (in NOVA or Maryland) than they can substitute to buying gas outside of the U.S. 

\end{solution}

\question (2018Q5) The U.S. Supreme Court has ruled that a proper test for deciding whether two  different products compete with each other (i.e., whether they belong in the same ``relevant market'') is the economic concept of cross-elasticity of demand.  The Court wrote, ``The . . . boundaries of a product market are determined by the cross-elasticity of demand between the product itself and substitutes for it.'' You have been asked: do carbonated soft drinks (such as Coca-Cola and Pepsi) compete with energy drinks (such as Red Bull and Monster)?  That is, do they belong in the same market?  How can the economist’s tool of cross-elasticity of demand help answer this question (as the Supreme Court suggests it might)? 

\begin{solution}
    The sign of the cross-price elasticity indicates whether goods are complements or substitutes: substitutes if positive; complements if negative. If goods compete in the same market, they will have a positive cross-price elasticity.
\end{solution}

\pagebreak

\question Define the following:
\begin{parts}
\part Pareto optimality
\begin{solution}
An allocation of resources in which no one can be made better off without making someone else worse off.
\end{solution}
\part Income elasticity of demand
\begin{solution} The percent change in quantity demanded in response to a percent change in income.
$$\frac{\%\Delta  Q}{\%\Delta  Income}$$
\end{solution}
\part Perfectly inelastic demand curve
\begin{solution}
A vertical line. Regardless of the price, the quantity demanded does not change. 
\end{solution}
\part Search or transaction costs

\begin{solution}
\\
Search costs: the cost of "shopping" or gathering information about a good. \\
Transaction costs: the cost of executing a transaction.
\end{solution}
\end{parts}

\end{questions}


\end{document}