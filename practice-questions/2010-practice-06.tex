\documentclass{exam}

\usepackage{graphicx}
\graphicspath{ {./images/} }

\lhead{ECON 2010}
\chead{Practice}
\lfoot{9/21/2023}
\rhead{Fall 2023}

% \printanswers
\noprintanswers

\begin{document}

\section{Practice Questions}

\begin{questions}

\question (2020E2Q1) You are talking to your friend Wellington about his new smartphone.  He says: ``it was only \$200, how could I afford not to buy it?"  But you recall that only a week ago you asked Wellington about going to see a movie, now that a few theaters are open for business.  You told him the tickets cost \$15 each and he said, ``Fifteen dollars!  I can’t afford that."  Using economic analysis, explain why Wellington’s behavior may be rational.

\begin{solution} The principle of rational choice requires that the marginal utility \textit{per dollar spent} is equal across goods. If Wellington's marginal utility from a smartphone is high and his marginal utility from a movie is low, his behavior may be rational. \end{solution}

\question (2019E2Q4) Eli plans to buy snacks for a day trip to Lake Anna. His two favorite snacks are trail mix (\$2 each bag) and apples (\$1 each). Using his knowledge of economics Eli writes down the following tables which show the utility he would receive from different quantities of each snack. 

\begin{parts}

\part Fill out the columns showing the marginal utility and the marginal utility per dollar for bags of trail mix and for apples.

\begin{table}[h]
\centering
\begin{tabular}{|c|c|c|c|c|}
    \hline
    \textbf{Trail Mix} & $U_{Trail Mix}$ & $MU_{Trail Mix}$ & $MU_{Trail Mix} / P_{Trail Mix}$ \\
    \hline
    0 & 0 & ------ & ------ \\
    1 & 20 &  &  \\
    2 & 36 &  &  \\
    3 & 48 &  &  \\
    4 & 56 &  &  \\
    5 & 60 &  &  \\
    \hline
\end{tabular}
\end{table}

\begin{table}[h]
\centering
\begin{tabular}{|c|c|c|c|c|}
    \hline
    \textbf{Apples} & $U_{Apples}$ & $MU_{Apples}$ & $MU_{Apples} /P_{Apples}$ \\
    \hline
    0 & 0 & ------ & ------ \\
    1 & 11 &  &  \\
    2 & 21 &  &  \\
    3 & 30 &  &  \\
    4 & 38 &  &  \\
    5 & 45 &  &  \\
    \hline
\end{tabular}
\end{table}

\begin{solution}

\begin{minipage}{\linewidth}
\centering
\begin{tabular}{|c|c|c|c|c|}
    \hline
    \textbf{Trail Mix} & $U_{Trail Mix}$ & $MU_{Trail Mix}$ & $MU_{Trail Mix} / P_{Trail Mix}$ \\
    \hline
    0 & 0 & ------ & ------ \\
    1 & 20 & 20 & 10 \\
    2 & 36 & 16 & 8 \\
    3 & 48 & 12 & 6 \\
    4 & 56 & 8 & 4 \\
    5 & 60 & 4 & 2 \\
    \hline
\end{tabular}

\vspace{1em}

\begin{tabular}{|c|c|c|c|c|}
    \hline
    \textbf{Apples} & $U_{Apples}$ & $MU_{Apples}$ & $MU_{Apples} /P_{Apples}$ \\
    \hline
    0 & 0 & ------ & ------ \\
    1 & 11 & 11 & 11 \\
    2 & 21 & 10 & 10 \\
    3 & 30 & 9 & 9 \\
    4 & 38 & 8 & 8 \\
    5 & 45 & 7 & 7 \\
    \hline
\end{tabular}
\end{minipage}
\end{solution}

\part If Eli has \$8 and his total utility is the sum of the utils from trail mix and the utils from apples, ($TU = U_{Trail Mix} + U_{Apples}$) how many apples and bags of trail mix would Eli buy? 

\begin{solution}
2 bags of trail mix and 4 apples
\end{solution}

\end{parts}

\end{questions}

\end{document}