%%%%%%%%%%%%%%%%%%%%%%%%%%%%%%%%%%%%%%%%%%%%%%%%%%%%%%%%%%%%%%%
%
% Welcome to Overleaf --- just edit your LaTeX on the left,
% and we'll compile it for you on the right. If you open the
% 'Share' menu, you can invite other users to edit at the same
% time. See www.overleaf.com/learn for more info. Enjoy!
%
%%%%%%%%%%%%%%%%%%%%%%%%%%%%%%%%%%%%%%%%%%%%%%%%%%%%%%%%%%%%%%%
\documentclass{exam}

\lhead{ECON 2010}
\chead{Practice}
\lfoot{8/31/2023}
\rhead{Fall 2023}

% \printanswers
\noprintanswers

\begin{document}

\section{Practice Questions}

\begin{questions}

\question (2022Q2). Three fourth year students at UVA rent an apartment together and decide to buy a couch to help furnish the place. In answering the questions below, assume the demand and supply model describes the market for couches in the U.S. economy.

\begin{parts}
    \part The market for couches initially is in equilibrium.  In the space below, draw a demand and supply curve to portray this market.  Label all curves, both axes, and the market equilibrium.
    \begin{solution}
        A graph with $P$ and $Q$ on the axes, $D$ and $S$ curves, and labels for $P^*$ and $Q^*$.
    \end{solution}

    \part What is meant by the market for couches being ``in equilibrium"?
    \begin{solution}
        Quantity supplied equals quantity demanded; the market is ``at rest''; the price at which the market clears.
    \end{solution}

    \part Assume that end tables are a complementary good to couches. What will happen to the demand for couches if there is a significant increase in the price of end tables? Explain why.
    \begin{solution}
        Demand for couches decreases.\\
        
        As the price of end tables increases, people buy fewer end tables, and since they usually buy end tables and couches together, they demand fewer couches.
    \end{solution}

    \part Assume that chairs are a substitute for couches. What will happen to the demand for couches if there is a significant drop in the price of chairs? Explain why.
    \begin{solution}
        Demand for couches decreases.\\

        As the price of chairs decreases, people choose to buy chairs instead of couches.
    \end{solution}

    \part Assume the equilibrium price of couches is \$500.  If the government mandates that couches cannot sell for more than \$300, what will be the likely result for buyers in the market for couches?
    \begin{solution}
        A shortage.
    \end{solution}

    \part Would an increase in demand exacerbate or reduce the administrative problem of “pegging” the price at \$300?
    \begin{solution}
        Exacerbate.
    \end{solution}

    \part Would an increase in supply exacerbate or reduce the administrative problem of “pegging” the price at \$300? 
    \begin{solution}
        Reduce.
    \end{solution}

    \part If the price of upholstery fabric and padding used in making couches were to fall, would this increase or decrease the supply of couches?
    \begin{solution}
        Increase.
    \end{solution}

    \part If couches were an inferior good, what would happen in the market for couches (in terms of price and quantity) if the income of prospective couch buyers were to drop significantly?
    \begin{solution}
        Both price and quantity would increase.
    \end{solution}

\end{parts}

\question (2022Q3) According to the July 23, 2022 issue of the Wall Street Journal, Matthew Taffer of Richmond buys and sells (or ``flips") sneakers - like the Air Jordan 1 Retro High and Nike Dunk Low White Black. In some circles, Mr. Taffer is called a ``Sneakerhead." In Econ 201, we call him an arbitrageur. Mr. Taffer buys the shoes new from the manufacturer, holds them in inventory hoping their price will go up, and then resells them to others. According to the article, this business model has fallen on tough times. The title of the article is ``The Crash of a Shoe Market." 

\begin{parts}
    \part Using the demand and supply model, identify two economic variables on the demand side of
    the market that, if they were to change, would help Mr. Taffer's business.
    \begin{solution}
            A change in tastes; an increase in consumer income; an increase in the number of consumers; an increase in the price of a substitute good; a decrease in the price of a complement good; reducing taxes or increasing subsidies; improving expectations about the future
    \end{solution}
    \part Identify two variables on the demand side of the market that Mr. Taffer would not welcome.
    \begin{solution}
        The opposite of part (b).
    \end{solution}
    \part Mr. Taffer does not have a ``brick and mortar" store; he operates out of his home; and he does not manufacture the shoes he sells. From an accounting perspective, apart from buying the shoes, it seems as though the cost of operating his business is near zero. From an economic perspective, what costs does he incur in operating his business?
    \begin{solution}
        Opportunity costs: the value of his time; the value of the space in his home used to store the shoes; the time-value of his capital that is tied up in inventory.
    \end{solution}
\end{parts}

\end{questions}


\end{document}
