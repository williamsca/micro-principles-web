%%%%%%%%%%%%%%%%%%%%%%%%%%%%%%%%%%%%%%%%%%%%%%%%%%%%%%%%%%%%%%%
%
% Welcome to Overleaf --- just edit your LaTeX on the left,
% and we'll compile it for you on the right. If you open the
% 'Share' menu, you can invite other users to edit at the same
% time. See www.overleaf.com/learn for more info. Enjoy!
%
%%%%%%%%%%%%%%%%%%%%%%%%%%%%%%%%%%%%%%%%%%%%%%%%%%%%%%%%%%%%%%%
\documentclass{exam}

\usepackage{graphicx}

\lhead{ECON 2010}
\chead{Problem Set \#4}
\lfoot{}
\rhead{Fall 2022}

\printanswers
% \noprintanswers

\begin{document}

\section{Graded Questions}

\begin{questions}

\question Politicians occasionally claim that the government should be run ``like a business." For this question, assume that they are correct. Suppose that the government is choosing whether to dedicate a new national park. Maintaining the new park would cost the government \$60,000 annually in wages and \$25,000 in supplies, including building materials, vehicles, and park ranger equipment. Revenues from entrance passes to the park are expected to total \$100,000. The prospective park would occupy lands which currently yield the government \$20,000 in rent from logging companies. These rents would be lost if the land were dedicated as a national park.

\begin{parts}
\part Calculate the accounting and economic profits. Based on these figures, should the government dedicate the park?

\part If the government chooses to dedicate the park, the value of nearby property will appreciate by an amount equivalent in value to \$30,000 per year. Should the government consider the appreciation of local property values when deciding whether to dedicate the park? Defend your view.

\end{parts}

% \question A legislator proposes that the government should build a post office within 50 miles of every U.S. resident. The value of this policy to rural households is estimated to be \$1.5 million dollars. Under what conditions would you support such a policy?


\question Consider a firm operating in a perfectly competitive industry. Sketch the firm's demand, average total and variable costs, and marginal cost curves under the assumption that the firm is earning economic profits.

\begin{parts}
\part Explain how your graph would change if the firm had no fixed costs.
\part Identify, on the horizontal axis, the firm's chosen output as $Q^*$.
\part Identify the firm's profits  on the graph.
\part Identify, on the vertical axis, the price below which the firm would shut down in the short-run as $P^{s}$ 
\part Identify, on the vertical axis, the price below which the firm will exit the market in the long-run as $P^e$.

\part Explain how the demand curve faced by this firm will change if more firms were to enter the market.

\end{parts}

\question Suppose that Harley spends his monthly disposable income of \$1,500 on trips to Europe, which cost \$500 each, and on home improvement projects, which cost \$250 per project.
\begin{parts}
\part Suppose that Harley takes one trip to Europe and performs two home projects. Is this consistent with the principle of rational choice?

\part Now suppose that Harley spends all of his income on trips to Europe. What does this choice imply about the marginal utilities of each product?
\end{parts}

\question Complete the following cost table. (You may find p. 147 of the e-textbook helpful).
\vspace{1em}

\begin{large}
\begin{tabular}{ |l|r|r|r|r|r|r|r| } 
\hline
Output & FC & VC & TC & AFC & AVC & ATC & MC \\
 \hline
0 & \$2 & \$0 & \$2 & \$--- & \$--- & \$--- & \$--- \\ \hline
 1 & 2 &3 & 5 &2 & 3 & 5 & 3\\ \hline
 2 & 2 &5 & & & & & \\ \hline
 3 & 2 &6 & & & & & \\ \hline
 4 & 2 &8 & & & & & \\ \hline
 5 & 2 &10 & & & & & \\ \hline
\end{tabular}
\end{large}

\end{questions}

\section{Extra Practice (ungraded)}

\begin{questions}

\question Sketch a monopolist's demand and marginal revenue curves. Can you draw average and marginal cost curves so that the monopolist earns zero profits?

\end{questions}

\end{document}