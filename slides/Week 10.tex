
%%%%%%%%%%%%%%%%%%%%%%%%%%%%%%%%%%%%%%%%%
% Focus Beamer Presentation
% LaTeX Template
% Version 1.0 (8/8/18)
%
% This template has been downloaded from:
% http://www.LaTeXTemplates.com
%
% Original author:
% Pasquale Africa (https://github.com/elauksap/focus-beamertheme) with modifications by 
% Vel (vel@LaTeXTemplates.com)
%
% Template license:
% GNU GPL v3.0 License
%
% Important note:
% The bibliography/references need to be compiled with bibtex.
%
%%%%%%%%%%%%%%%%%%%%%%%%%%%%%%%%%%%%%%%%%

%----------------------------------------------------------------------------------------
%	PACKAGES AND OTHER DOCUMENT CONFIGURATIONS
%----------------------------------------------------------------------------------------

\documentclass{beamer}
\usepackage{enumitem}
\beamertemplatenavigationsymbolsempty % suppress navigation bar

\usetheme{default} % Use the Focus theme supplied with the template
% Add option [numbering=none] to disable the footer progress bar
% Add option [numbering=fullbar] to show the footer progress bar as always full with a slide count

% Uncomment to enable the ice-blue theme
%\definecolor{main}{RGB}{92, 138, 168}
%\definecolor{background}{RGB}{240, 247, 255}

%------------------------------------------------

\usepackage{booktabs} % Required for better table rules

\begin{document}

\begin{frame}{Bonus Questions}
Consider a yoghurt firm facing perfectly elastic demand at $P = $ \$1.2 per unit. The firm's $AC$ is \$1.4 and its $AVC$ \$1.0 per unit. If there is no prospect of change, the firm should:

\begin{enumerate}[label=\alph*)]
\item Shut down immediately
\item Continue to produce in the LR (where $MC = MR$) but not in the SR
\item Raise its price to at least \$1.8 to cover its costs
\item Continue to produce output in the SR (where $AC = MR$) but not in the LR
\item Continue to produce in the SR (where $MC = MR$) but not in the LR
\end{enumerate}
\end{frame}

\begin{frame}{Bonus Questions}
A profit maximizing firm that is a ``price maker'' will:
\begin{enumerate}[label=\alph*)]
\item Select a price that maximizes the difference between $P$ and $AC$
\item Select a price that is higher and an output that is greater than a ``price taker'' would choose
\item Select an output where $MC=MR$ and charge the highest price on the demand curve at that output
\item Select an output where $MC = MR$ and charge a price where $P = MC$
\item Charge a price that is the highest price on the demand curve and produce that output
\end{enumerate}

\end{frame}

\begin{frame}{Price Discrimination}
\includegraphics[width = \textwidth]{images/tinder.png}
\end{frame}

\begin{frame}{Price Discrimination}
\centering
\includegraphics[width = .8\textwidth, height = .8\textheight]{images/seniorcoffee.jpg}
\end{frame}

\end{document}
