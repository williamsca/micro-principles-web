
%%%%%%%%%%%%%%%%%%%%%%%%%%%%%%%%%%%%%%%%%
% Focus Beamer Presentation
% LaTeX Template
% Version 1.0 (8/8/18)
%
% This template has been downloaded from:
% http://www.LaTeXTemplates.com
%
% Original author:
% Pasquale Africa (https://github.com/elauksap/focus-beamertheme) with modifications by 
% Vel (vel@LaTeXTemplates.com)
%
% Template license:
% GNU GPL v3.0 License
%
% Important note:
% The bibliography/references need to be compiled with bibtex.
%
%%%%%%%%%%%%%%%%%%%%%%%%%%%%%%%%%%%%%%%%%

%----------------------------------------------------------------------------------------
%	PACKAGES AND OTHER DOCUMENT CONFIGURATIONS
%----------------------------------------------------------------------------------------

\documentclass{beamer}
\usepackage{enumitem}
\beamertemplatenavigationsymbolsempty % suppress navigation bar

\usetheme{default} % Use the Focus theme supplied with the template
% Add option [numbering=none] to disable the footer progress bar
% Add option [numbering=fullbar] to show the footer progress bar as always full with a slide count

% Uncomment to enable the ice-blue theme
%\definecolor{main}{RGB}{92, 138, 168}
%\definecolor{background}{RGB}{240, 247, 255}

%------------------------------------------------

\usepackage{booktabs} % Required for better table rules

\begin{document}

\begin{frame}{Questions}
(Ch. 8.7) There's a gas shortage in Gasland. You're presented with two proposals that will achieve the same level of reduction in the use  of gas. Proposal A would force everybody to reduce their gas consumption by 5\%. Proposal B would impost a 50-cent tax on the consumption of a gallon of gas, which would also achieve a 5 \% reduction. Consumers can be divided into two groups: one group whose demand is elastic, and another whose demand is inelastic.
\begin{enumerate}[label=\alph*)]
\item How will the proposals affect each group?
\item Which group would support the regulatory policy (A)? Which would support the tax policy (B)?
\end{enumerate}
\end{frame}

\begin{frame}{Questions}
(Ch. 8.8) Economists studied the effect of Charlottesville, VA moving from charging a flat fee for garbage collection to charging \$0.80 per 32-gallon bag and found the following: the \textbf{weight} of garbage collected fell by 14\%; the \textbf{volume} of garbage collected fell by 37\%; and the weight of recycling rose by 16\%.
\begin{enumerate}[label=\alph*)]
\item Why did recycling increase and garbage collection decrease?
\item Why did the weight fall by less than the volume?
\item Demonstrate, using supply and demand curves, the effect of the change in pricing on the volume of garbage collected.
\end{enumerate}
\end{frame}


\end{document}
